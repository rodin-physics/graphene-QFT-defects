\documentclass[aps, prb, superscriptaddress, preprint, floatfix]{revtex4-1}
\usepackage{graphicx}% Include figure files
\usepackage{dcolumn}% Align table columns on decimal point
\usepackage{bm}% bold math
\usepackage{amsmath}

\usepackage{color}
\bibliographystyle{apsrev4-1}
\usepackage{bm,graphicx,hyperref}
\hypersetup{%
  breaklinks = {true},
  citecolor = {blue},
  colorlinks = {true},
  linkcolor = {red},
}

\begin{document}

\section{Non-orthogonal States}

In the tight-binding approximation, one commonly neglects the orbital overlap integral (treating the states in the system as orthogonal). While this assumption is generally well-motivated on physical grounds, the finite overlap does have an impact on the quantitative results. We start by providing a quick guide to help one write down the action for non-orthogonal states.

A general bilinear form of a Hamiltonian describing coupling between not-necessarily-orthogonal states is
%
\begin{equation}
	\hat{H} = \sum_{ab} c_a^\dagger H_{ab} c_b = \mathbf{c}^\dagger H \mathbf{c}\,,
\end{equation}
%
where $a$ and $b$ label the individual states. Allowing for the overlap, the eigenstates $\Psi$ of the system are found by solving the generalized eigenvalue problem
%
\begin{equation}
	H\Psi = E P\Psi\,,
\end{equation}
%
where $P$ is the overlap matrix. One can transform this Hamiltonian as follows
%
\begin{equation}
	H\Psi= E_\mathbf{q}P^{1/2} P^{1/2} \Psi
	\rightarrow 
	P^{-1/2}HP^{-1/2}P^{1/2}\Psi = E P^{1/2} \Psi\,.
\end{equation}
%
Hence, it is useful to define $\mathbf{c} = P^{-1/2}\tilde{\mathbf{c}}$ and write the second-quantized Hamiltonian as
%
\begin{align}
    \hat{H} &= \tilde{\mathbf{c}}^\dagger P^{-1/2}
    H
    P^{-1/2}\tilde{\mathbf{c}}
    \,.
\end{align}
%

Next, one can translate the Hamiltonian into the Matsubara-frequency action
%
\begin{align}
    S &= \sum_{\omega_n} \bar\psi_{\omega_n}
    \left[\left(-i\omega_n-\mu\right)
    + 
    P^{-1/2}
    H
    P^{-1/2}
    \right]
    \psi_{\omega_n}
    \nonumber
    \\
    &= \sum_{\omega_n} \bar\psi_{\omega_n}
    P^{-1/2}
    \left[P\left(-i\omega_n-\mu\right)
    + 
    H
    \right]
    P^{-1/2}\psi_{\omega_n}
    \nonumber
    \\
    &= \sum_{\omega_n} \bar\phi_{\omega_n}
    \left[P\left(-i\omega_n-\mu\right)
    + 
    H
    \right]
    \phi_{\omega_n}
    \,,
\end{align}
%
where $\phi_{\omega_n}$ is a vector of Grassmann numbers corresponding to the original operators $\mathbf{c}$. Hence, one can see that including the overlap amounts to adding $P_{ab}(-i\omega_n - \mu)$ to $H_{ab}$.

It is important to note that because of the finite overlap, the operators $c^\dagger_a$ and $c_b$ do not obey the familiar commutation relation and, therefore, do not form a canonical pair:
%
\begin{align}
	\langle 0|\{c_a^\dagger, c_b\}|0\rangle 
	&= 
	\langle 0|c_a^\dagger c_b + c_b c_a^\dagger|0\rangle
	= 
	\langle 0|c_a^\dagger c_b|0\rangle
	+
	\langle 0|c_b c_a^\dagger|0\rangle
	\nonumber
	\\
	&= 
	\langle 0|c_b c_a^\dagger|0\rangle = \langle b| a\rangle = P_{ba}.
\end{align}
%

On the other hand, we have
%
\begin{align}
	\langle 0|\{c_a^\dagger, \sum_{b}P_{jb}^{-1} c_b\}|0\rangle 
	&= 
	\langle 0|c_a^\dagger \sum_{b}P_{jb}^{-1} c_b + \sum_{b}P_{jb}^{-1} c_b c_a^\dagger|0\rangle
	\nonumber
	\\
	&= 
	\langle 0|\sum_{b}P_{jb}^{-1} c_b c_a^\dagger|0\rangle  = \sum_{b}P_{jb}^{-1}P_{ba} = \delta_{aj}.
\end{align}
%

\section{Prototypical System}




When the overlap is included, the two-band eigenvalues are found by solving the generalized eigenvalue problem

%
\begin{align}
	H^G_{0,\mathbf{q}}\Psi_\mathbf{q} = E_\mathbf{q}P_\mathbf{q} \Psi_\mathbf{q}\,,
\end{align}
%
where $P_\mathbf{q}$ is a $2\times 2$ overlap matrix. One can transform this Hamiltonian as follows

%
\begin{align}
	&H^G_{0,\mathbf{q}}\Psi_\mathbf{q} = E_\mathbf{q}P^{1/2}_\mathbf{q} P^{1/2}_\mathbf{q} \Psi_\mathbf{q}
	\nonumber
	\\
	\rightarrow &P^{-1/2}_\mathbf{q}H^G_{0,\mathbf{q}}P^{-1/2}_\mathbf{q}P^{1/2}_\mathbf{q}\Psi_\mathbf{q} = E_\mathbf{q} P^{1/2}_\mathbf{q} \Psi_\mathbf{q}\,.
\end{align}
%

Hence, it is useful to define $c_\mathbf{q} = P_\mathbf{q}^{-1/2}\tilde{c}_\mathbf{q}$ to write the second-quantized Hamiltonian as

%
\begin{align}
    \hat{H} &= \sum_{\mathbf{q}} \tilde{c}_\mathbf{q}^\dagger P_\mathbf{q}^{-1/2}
    H_{0,\mathbf{q}}^G 
    P_\mathbf{q}^{-1/2}\tilde{c}_\mathbf{q}
    \nonumber
    \\
    &+
    \frac{1}{N}\sum_{\mathbf{qq}'} \tilde{c}_\mathbf{q}^\dagger P_\mathbf{q}^{-1/2} \Theta_{\mathbf{q}}^\dagger \tilde{\Delta} \Theta_{\mathbf{q}'}P_{\mathbf{q}'}^{-1/2}\tilde{c}_{\mathbf{q}'}
    \,.
    \label{eqn:H_QFT_2}
\end{align}
%

Next, one can translate the Hamiltonian into the imaginary-time action
%
\begin{align}
    S &= \sum_{\omega_n\mathbf{qq}'} \bar\psi_{\omega_n\mathbf{q}}
    \left[\left(-i\omega_n-\mu\right) \delta_{\mathbf{qq}'}
    + 
    P_\mathbf{q}^{-1/2}
    H^G_{\mathbf{qq}'} 
    P_{\mathbf{q}'}^{-1/2}
    \right]
    \psi_{\omega_n\mathbf{q}'}\,.
    \label{eqn:S}
\end{align}
%
Note that we have combined the $\mathbf{q}$-diagonal and non-diagonal portions of the graphene Hamiltonian into $ H^G_{\mathbf{qq}'}$. Finally, we change the graphene fields back to the original orbitals to get
%
\begin{align}
    S &= \sum_{\omega_n\mathbf{qq}'} \bar\phi_{\omega_n\mathbf{q}}
    \overbrace{
    \left[P_\mathbf{q}\left(-i\omega_n-\mu\right) \delta_{\mathbf{qq}'}
    + 
    H^G_{\mathbf{qq}'} \right]}^{-G^{-1}_{i\omega_n + \mu, \mathbf{qq}'}}
    \phi_{\omega_n\mathbf{q}'}\,.
    \label{eqn:S_2}
\end{align}
%
%Importantly, there is a Jacobian associated 

Defining a pristine graphene Green's function $G_{z}^0 = \left(zP - H^G_0\right)^{-1}$ leads to

%
\begin{align}
	G_{z} &= \left[\left(G_{z}^0\right)^{-1} - \frac{1}{N} \Theta^\dagger \tilde{\Delta}  \Theta\right]^{-1}
	\nonumber
	\\
	&=
	G_{z}^0
	+
	\frac{1}{N} G_{z}^0\Theta^\dagger \tilde{\Delta}
	\left( 1
	-
	 \boldsymbol{\Xi}_{z} \tilde{\Delta}
	\right)^{-1} \Theta G_{z}^0\,,	
\end{align}
%
where $\boldsymbol{\Xi}^0_{z} = \Theta G_{z}^0\Theta^\dagger / N$ with entries $\left[\boldsymbol{\Xi}^0_{z}\right]_{\mathbf{R}_j\mathbf{R}_k} = \Xi_{z}^0\left(\mathbf{R}_j - \mathbf{R}_k\right)$, $\Theta$ as a row vector of $\Theta_\mathbf{q}$, and

%
\begin{equation}
    \Xi_z^0\left(\mathbf{R}\right) = \frac{1}{N}\sum_\mathbf{q}G_{z\mathbf{q}}^0
    e^{i \mathbf{R} \cdot\mathbf{q}}\,.
    \label{eqn:Xi}
\end{equation}
%


	
\section{Prototype System}



To model graphene coupled to external electronic states and influenced by an external potential, it is beneficial to start with the following generic Hamiltonian
%
\begin{align}
	\hat{H} &= \sum_{ab} c^\dagger_a \left(H_{ab} +\Delta_{ab}\right)c_b + \sum_{ab} d^\dagger_a h_{ab} d_b+ \sum_{ab} d^\dagger_a V_{ab} c_b + c^\dagger_b V_{ab}^* d_a
	\,.
\end{align}
%
Here, we have introduced two fermionic systems with the corresponding operators $c$ and $d$, governed by the Hamiltonians $H$ and $h$, respectively. The subscripts $a$ and $b$ label the individual states in these systems, while $V_{ab}$ represent the coupling between them. The term $\Delta_{ab}$ describes a general two-operator perturbation of one of the systems.

The expression can be made more compact by introducing $\mathbf{c}$ and $\mathbf{d}$ as column vectors of $c_a$ and $d_a$, respectively:
%
\begin{align}
	\hat{H} &= 	\mathbf{c}^\dagger \left(H+\Delta\right)\mathbf{c}
	+ 	\mathbf{d}^\dagger h\mathbf{d}
	+ \mathbf{d}^\dagger V \mathbf{c} +  \mathbf{c}^\dagger V^\dagger \mathbf{d}
	\,.
\end{align}
%

%
\begin{align}
	\hat{H} &= \sum_{ab} c^\dagger_a H_{ab} c_b 
	+ 
	\sum_{aa'bb'}U_{aa'bb'} c_a^\dagger c_b^\dagger c_{b'} c_{a'}
	\nonumber
	\\
	&+ \sum_{ab} d^\dagger_a h_{ab} d_b
	 +
	  \sum_{aa'bb'}u_{aa'bb'} d_a^\dagger d_b^\dagger d_{b'} d_{a'}
	\nonumber
	\\
	&+ \sum_{ab} d^\dagger_a V_{ab} c_b + c^\dagger_b V_{ab}^* d_a
	+ 
	\sum_{aa'bb'}W_{aa'bb'} c_a^\dagger d_b^\dagger d_{b'} c_{a'}\,,
\end{align}
%
We also include the two-body interaction terms $U$ and $u$ for each system, as well as a two-body interaction $W$ between the systems. The expression can be made more compact by introducing $\mathbf{c}$ and $\mathbf{d}$ as column vectors of $c_a$ and $d_a$, respectively:
%
\begin{align}
	\hat{H} &= 	\mathbf{c}^\dagger H\mathbf{c}
	+ 
	\sum_{aa'bb'}U_{aa'bb'} c_a^\dagger c_b^\dagger c_{b'} c_{a'}
	\nonumber
	\\
	&+ 	\mathbf{d}^\dagger h\mathbf{d}
	 +
	  \sum_{aa'bb'}u_{aa'bb'} d_a^\dagger d_b^\dagger d_{b'} d_{a'}
	\nonumber
	\\
	&+ \mathbf{d}^\dagger V \mathbf{c} +  \mathbf{c}^\dagger V^\dagger \mathbf{d}
	+ 
	\sum_{aa'bb'}W_{aa'bb'} c_a^\dagger d_b^\dagger d_{b'} c_{a'}\,.
\end{align}
%

Because the Hamiltonian operator is normal-ordered, we can transcribe it into the imaginary-time action
%
\begin{align}
	S &= \int_0^\beta d\tau
	\Bigg\{\bar\psi(\tau) \left[\partial_\tau - \mu + H\right]\psi(\tau)
	+ 
	\sum_{aa'bb'}U_{aa'bb'} \bar\psi_a(\tau)\bar\psi_b(\tau)\psi_{b'}(\tau)\psi_{a'}(\tau) 
	\nonumber
	\\
	&+ \bar\phi(\tau)\left[\partial_\tau - \mu + h\right]\phi(\tau)
	 +
	  \sum_{aa'bb'}u_{aa'bb'} \bar\phi_a(\tau)\bar\phi_b(\tau)\phi_{b'}(\tau)\phi_{a'}(\tau) 
	\nonumber
	\\
	&+ \bar\phi(\tau) V \psi(\tau) +  \bar\psi(\tau) V^\dagger \phi(\tau)
	+ 
	\sum_{aa'bb'}W_{aa'bb'} \bar\psi_a(\tau)\bar\phi_b(\tau)\phi_{b'}(\tau)\psi_{a'}(\tau) \Bigg\}\,.
\end{align}
%

To describe an infinitely-large graphene system with multiple defects, we start with the standard tight-binding $p_z$ orbital Hamiltonian including only the nearest-neighbor hopping with the parameter $t$
%
\begin{equation}
	\hat{H}_{0}^G 
	= 
	\sum_{\sigma,\mathbf{q}}
	\begin{pmatrix}
		a^\dagger_{\sigma,\mathbf{q}} & b^\dagger_{\sigma,\mathbf{q}}
	\end{pmatrix} 
	\begin{pmatrix}
       0&-tf_\mathbf{q}
       \\
       -tf_\mathbf{q}^*&0
   \end{pmatrix} 
   \begin{pmatrix}
		a_{\sigma,\mathbf{q}} \\ b_{\sigma,\mathbf{q}}
	\end{pmatrix} 
	= 
	\sum_{\sigma,\mathbf{q}}
	c^\dagger_{\sigma,\mathbf{q}}
	H_{0,\mathbf{q}}^G 
	c_{\sigma,\mathbf{q}} 
	= \mathbf{c}^\dagger H^G_0 \mathbf{c}\,.
	\label{eqn:H0}
\end{equation}
%
Here, $\mathbf{c}$ is a column vector of $c_{\sigma,\mathbf{q}}$ and $H_0^G$ is the corresponding matrix. Note that, to keep the problem general, the operators are spin-dependent, as can be seen from the $\sigma$ subscript.

Next, we augment the Hamiltonian by introducing external electronic states that can be coupled to each other, as well as to individual graphene orbitals
%
\begin{align}
	\hat{H}_S 
	&=
	\sum_{\sigma\sigma', jk} g_{\sigma, j}^\dagger h^{\sigma\sigma'}_{jk} g_{ \sigma',k} + 
	\sum_{\sigma\sigma',k,\mathbf{R}}\left[c^\dagger_{\sigma,\mathbf{R}} V^{\sigma\sigma'}_{\mathbf{R},k}g_{\sigma',k} + g_{\sigma',k}^\dagger \left(V^{\sigma\sigma'}_{\mathbf{R},k}\right)^\dagger c_{\sigma,\mathbf{R}}\right]
	\nonumber
	\\
	&=
	\mathbf{g}^\dagger h\mathbf{g}
	 + 
	 \frac{1}{\sqrt{N}}\mathbf{c}^\dagger \Theta^\dagger V\mathbf{g} 
	 + 
	 \frac{1}{\sqrt{N}}\mathbf{g}^\dagger V^\dagger \Theta \mathbf{c}
	\,,
\end{align}
%
where $\mathbf{g}$ is a column vector of $g_{\sigma,k}$ and $V^{\sigma\sigma'}_{\mathbf{R},k}$ is a $2\times 1$ matrix. The summation over $\mathbf{R}$ includes all the unit cells in the system, even if the coupling between a particular unit cell and the external electronic states is zero. In the last line, we used $c^\dagger_{\sigma,\mathbf{R}} = N^{-1/2}\sum_\mathbf{q}c^\dagger_{\sigma,\mathbf{q}} e^{-i\mathbf{R}\cdot\mathbf{q}}$, where $N$ is the number of unit cells in the system, to define the Fourier transform matrix $\Theta$.

Finally, we include a two-operator energy term that modifies the orbital energy and the hopping between two atoms, as well as a four-operator Coulomb electron-electron interaction term:
%
\begin{align}
	\hat{H}_{e} &= 
	\sum_{\sigma\sigma',\mathbf{RR}'} c^\dagger_{\sigma,\mathbf{R}}\Delta_{\mathbf{RR}'}^{\sigma\sigma'}c_{\sigma',\mathbf{R}'}
	+
	\sum_{\sigma\sigma',\mathbf{RR}'}\sum_{jk}
	U_{\mathbf{RR}'}^{jk}
	c_{k,\sigma,\mathbf{R}}^\dagger
	c_{j,\sigma',\mathbf{R}'}^\dagger
	c_{j,\sigma',\mathbf{R}'}
	c_{k,\sigma,\mathbf{R}}
	\nonumber
	\\
	&= 
	\frac{1}{N}\mathbf{c}^\dagger \Theta^\dagger \Delta\Theta\mathbf{c}
	+
	\frac{1}{N^2}\sum_{\sigma\sigma',\mathbf{RR}'}\sum_{jk}
	U_{\mathbf{RR}'}^{jk}
	\left(\mathbf{c}^\dagger\Theta^\dagger \right)_{k,\sigma,\mathbf{R}}
	\left(\mathbf{c}^\dagger\Theta^\dagger \right)_{j,\sigma',\mathbf{R}'}
	\left(\Theta\mathbf{c}\right)_{j,\sigma',\mathbf{R}'}
	\left(\Theta\mathbf{c}\right)_{k,\sigma,\mathbf{R}}
	\nonumber
	\\
	&= 
	\frac{1}{N}\mathbf{c}^\dagger \Theta^\dagger \Delta\Theta\mathbf{c}
	+
	\frac{1}{N^2}\sum_{AB}
	U_{AB}	\left(\mathbf{c}^\dagger\Theta^\dagger \right)_{A}
	\left(\mathbf{c}^\dagger\Theta^\dagger \right)_{B}
	\left(\Theta\mathbf{c}\right)_{B}
	\left(\Theta\mathbf{c}\right)_{A}\,,
\end{align}
%
where $j$ and $k$ label the sublattice. In the last line, we combine all the state labels into $A$ and $B$ for brevity.

\section{Action}

Following the standard QFT approach, the Hamiltonian can be written as the imaginary-time action
%
\begin{align}
	S &= \int d\tau
	\Bigg\{
	\sum_{\sigma,\mathbf{q}}
	\bar\psi_{\sigma,\mathbf{q}}(\tau)
	\left[\partial_\tau -\mu + H_{0,\mathbf{q}}^G \right]
	\psi_{\sigma,\mathbf{q}}(\tau)
	+
	\sum_{\sigma, k} \bar\gamma_{\sigma, k}(\tau) \left[\partial_\tau - \mu + \varepsilon_{\sigma,k}  \right]\gamma_{ \sigma,k}(\tau) 
	\nonumber
	\\
	&+ 
	\sum_{\sigma,k,\mathbf{R}}\left[ \bar\psi_{\sigma,\mathbf{R}} (\tau)V_{\mathbf{R},k}\gamma_{\sigma,k}(\tau) + \bar\gamma_{\sigma,k}(\tau) V_{\mathbf{R},k}^\dagger \psi_{\sigma,\mathbf{R}}(\tau)\right]
	\nonumber
	\\
	&+
	\sum_{\sigma\sigma',\mathbf{RR}'} \bar\psi_{\sigma,\mathbf{R}}(\tau)\Delta_{\mathbf{RR}'}^{\sigma\sigma'}\psi_{\sigma',\mathbf{R}'}(\tau)
	\nonumber
	\\
	&+
	\sum_{\sigma\sigma',\mathbf{RR}'}\sum_{ij}
	U_{\mathbf{RR}'}^{jk}
	\bar\psi_{k,\sigma,\mathbf{R}}(\tau)
	\bar\psi_{j,\sigma',\mathbf{R}'}(\tau)
	\psi_{j,\sigma',\mathbf{R}'}(\tau)
	\psi_{k,\sigma,\mathbf{R}}(\tau)
	\Bigg\}\,.
\end{align}
% 



the following Hamiltonian:
%
\begin{align}
    \hat{H} &= \sum_{\mathbf{q}} c^\dagger_{\mathbf{q}}
    \left(H_{0,\mathbf{q}}^G - \mu\right)
    c_{\mathbf{q}}
    +
    \sum_{ k} g_{ k}^\dagger \left(\varepsilon_k - \mu\right)  g_{ k}
    \nonumber
    \\
    &+
    \sum_{ jk} 
    \left[
    c^\dagger_{\mathbf{R}_j}I_j V_{j,k} g_{ k} 
    + 
    g_{ k}^\dagger \left(V_{j,k}\right)^* I_j^T c_{\mathbf{R}_j}
    \right]
    \nonumber
    \\
    &+  
    \sum_{jl} c_{\mathbf{R}_j}^\dagger I_j \Delta_{jl} I^T_lc_{\mathbf{R}_l}
    =
    \sum_{}
    \,.
    \label{eqn:H0_QFT}
\end{align}
%


%
\begin{table*}
  \centering
\begin{tabular}{ |c|m{10cm}| } 
 \hline
$H_{0,\mathbf{q}}^G$ & Pristine graphene Hamiltonian matrix
 \\
 \hline
$\mu$ & Chemical potential
\\ 
 \hline 
 $c^\dagger_{\mathbf{q}} = \begin{pmatrix} a^\dagger_{\mathbf{q}}&b^\dagger_{\mathbf{q}}\end{pmatrix}$ & Vector of the creation operators for the carbon $p_z$ orbitals for the two sublattices in momentum space. $c^\dagger_{\mathbf{R}}$ is the real-space counterpart
 \\ 
 \hline 
 $g^\dagger_k$ & Creation operator for the impurity state of energy $\varepsilon_k$
 \\
 \hline
 $\Delta_{jl}$ & Modified coupling/on-site energy for graphene
 \\ 
 \hline 
 $V_{j,k}$ & Atom $j$'s interaction strength with the impurity state $k$
 \\
\hline 
 $\Delta
 = \begin{pmatrix}
 	\Delta_{11} & \Delta_{12}& \cdots
 	\\
 	\Delta_{21} & \Delta_{22}& \cdots
 	\\
 	\vdots & \vdots & \ddots
 \end{pmatrix}$ & $M\times M$ matrix of couplings between carbon $p_z$ orbitals  \\ 
 \hline 
 $V= \begin{pmatrix}
 	V_{1,1} & V_{1,2}& \cdots
 	\\
 	V_{2,1} & V_{2,2}& \cdots
 	\\
 	\vdots & \vdots & \ddots
 \end{pmatrix}$ & $M\times K$ matrix
\\ 
 \hline
  $I_j^T = \begin{pmatrix} 1 & 0 \end{pmatrix}$ or $ \begin{pmatrix} 0 & 1 \end{pmatrix}$ & A vector for choosing the correct graphene sublattice
   \\
 \hline
\end{tabular}
  \caption{Table of quantities used in the Hamiltonian. Here, $K$ is the number of impurity states and $M$ is the number of graphene atoms affected by the defects.}
  \label{tab:Reference}
\end{table*}
%

The first line describes the pristine graphene system and the isolated impurities. The second line provides the coupling between the impurity states and graphene atoms at unit cells with coordinates $\mathbf{R}_j$. Importantly, the sum $j$ runs over \emph{all} the atoms impacted by the impurities, either by directly interacting with them or because the induced lattice deformation changes their coupling to other graphene atoms. Finally, the last line gives the perturbation of the graphene Hamiltonian due to the lattice deformation. As with the line above, the sum includes all the modified atoms.

Using  $c^\dagger_\mathbf{R} = N^{-1/2}\sum_\mathbf{q}c^\dagger_\mathbf{q} e^{-i\mathbf{R}\cdot\mathbf{q}}$, where $N$ is the number of unit cells in the system, one gets
%
\begin{equation}
    \sum_{j}
    c^\dagger_{\mathbf{R}_j} I_j V_{j,k}
    = 
    \frac{1}{\sqrt{N}}
    \sum_\mathbf{q}c^\dagger_{\mathbf{q}}
    \underbrace{\left(\sum_{j}e^{-i\mathbf{R}_j\cdot\mathbf{q}} I_j V_{j,k}\right)}_{\Theta_\mathbf{q}^\dagger \mathbf{I} V_{,k}}\,,
    \label{eqn:Coupling}
\end{equation}
%
where $\Theta_\mathbf{q}$ is a column vector of $\mathbf{1}_{2\times 2} e^{i\mathbf{R}_j\cdot\mathbf{q}}$ for all $\mathbf{R}_j$, $\mathbf{I}$ is a diagonal matrix of $I_j$, and $V_{,k}$ is a column vector of $V_{j,k}$. Similarly,
%
\begin{align}
	\sum_{jk} c_{\mathbf{R}_j}^\dagger I_j \Delta_{jk} I^T_kc_{\mathbf{R}_k}
	=\frac{1}{N}\sum_{\mathbf{qq}'} c_{\mathbf{q}}^\dagger \Theta_{\mathbf{q}}^\dagger\mathbf{I}\Delta \mathbf{I}^T\Theta_{\mathbf{q}'}c_{\mathbf{q}'}\,.
	\label{eqn:Delta}
\end{align}
%

\section{Partition Function}

Plugging Eqs.~\eqref{eqn:Coupling} and \eqref{eqn:Delta} into Eq.~\eqref{eqn:H_QFT}, one can translate the Hamiltonian into the imaginary-time action
%
\begin{align}
    S &= \sum_{\omega_n\mathbf{qq}'} \bar\psi_{\omega_n\mathbf{q}}
    \overbrace{
    \left[\left(-i\omega_n-\mu\right) \delta_{\mathbf{qq}'}
    + 
    H^G_{\mathbf{qq}'} \right]}^{-G^{-1}_{i\omega_n + \mu, \mathbf{qq}'}}
    \psi_{\omega_n\mathbf{q}'}
    \nonumber
    \\
    &
    +
    \sum_{\omega_n k}\bar\gamma_{\omega_n,k} 
    \overbrace{\left(-i\omega_n - \mu + \varepsilon_k \right)}^{-\Gamma_{0,i\omega_n + \mu, k}^{-1}} \gamma_{\omega_n,k}
    \nonumber
    \\
    &
    +
    \frac{1}{\sqrt{N}}\sum_{\omega_n k \mathbf{q}} 
    \left(
    \bar\psi_{\omega_n,\mathbf{q}}
    \Theta_\mathbf{q}^\dagger
    \mathbf{I}V_{,k}
    \gamma_{\omega_n,k}+
    \bar\gamma_{\omega_n,k} 
    V_{,k}^\dagger\mathbf{I}^T\Theta_\mathbf{q}
    \psi_{\omega_n,\mathbf{q}}
    \right)\,.
    \label{eqn:S}
\end{align}
%
Note that we have combined the $\mathbf{q}$-diagonal and non-diagonal portions of the graphene Hamiltonian into $ H^G_{\mathbf{qq}'}$. The quantity $\omega_n$ is the fermionic Matsubara frequency, and $\gamma$ and $\psi$ are Grassmann fields. Integrating $e^{-S}$ over all the fields gives the partition function
%
\begin{align}
    \mathcal{Z} &= 
    \prod_{\omega_n }
    \left|-\beta G^{-1}_{i\omega_n+\mu}\right|
    \left|-\beta\left(\Gamma^{-1}_{0,i\omega_n+\mu} - \frac{V^\dagger \mathbf{I}^T\Theta  G_{i\omega_n+\mu} \Theta^\dagger \mathbf{I} V}{N}\right)\right|
    \nonumber
    \\
    &= 
    \prod_{\omega_n }
    \left|-\beta \Gamma^{-1}_{0,i\omega_n+\mu}\right|
    \left|-\beta\left( G_{i\omega_n+\mu} ^{-1} -  \frac{\Theta^\dagger\mathbf{I}V\Gamma_{0,i\omega_n+\mu}V^\dagger \mathbf{I}^T\Theta  }{N}\right)\right|\,,
    \label{eqn:Z}
\end{align}
%
where $\Theta$ as a row vector of $\Theta_\mathbf{q}$. Defining a pristine graphene Green's function $G_{z}^0 = \left(z - H^G_0\right)^{-1}$ leads to
%
\begin{align}
	G_{z} &= \left[\left(G_{z}^0\right)^{-1} - \frac{1}{N} \Theta^\dagger\mathbf{I}\Delta \mathbf{I}^T\Theta\right]^{-1}
	=
	G_{z}^0
	+
	\frac{1}{N} G_{z}^0\Theta^\dagger\mathbf{I}\Delta
	\left( 1
	-
	\mathbf{I}^T\boldsymbol{\Xi}_{z}\mathbf{I}\Delta 
	\right)^{-1}\mathbf{I}^T\Theta G_{z}^0\,,	
\end{align}
%
where $\boldsymbol{\Xi}_{z} = \Theta G_{z}^0\Theta^\dagger / N$ with entries $\boldsymbol{\Xi}_{z}^{jk} = \Xi_{z}^{\mathbf{R}_j - \mathbf{R}_k}$ and
%
\begin{equation}
    \Xi_z^\mathbf{R} = \frac{1}{N}\sum_\mathbf{q}G_{z\mathbf{q}}^0
    e^{i \mathbf{R} \cdot\mathbf{q}}\,.
    \label{eqn:Xi}
\end{equation}
%
The derivation of $\Xi_z^\mathbf{R}$ is given below. $G_z$ is the graphene Green's function including the lattice deformation, but not the effects of the impurity states.

In the parentheses of the first line of Eq.~\eqref{eqn:Z}, we identify the inverse of the full impurity Green's function, denoted by $\Gamma^{-1}_{z}$:
%
\begin{align}
	\Gamma_{z} & = \left(
    \Gamma^{-1}_{0,z}
    - 
   V^\dagger\Lambda_{z}V
	\right)^{-1}
	=
	\Gamma_{0,z}
	+
	\Gamma_{0,z}
    V^\dagger\Lambda_{z}
	\left(
	1
     - V
    \Gamma_{0,z}
    V^T\Lambda_{z}
	\right)^{-1}V\Gamma_{0,z}
	\,,
    \label{eqn:Gamma}
    \\
    \Lambda_{z} &=
    \mathbf{I}^T\boldsymbol{\Xi}_{z}\mathbf{I}
	\left[1+
    \Delta
	\left( 1
	-
	\mathbf{I}^T\boldsymbol{\Xi}_{z}\mathbf{I}\Delta 
	\right)^{-1} \mathbf{I}^T\boldsymbol{\Xi}_{z}\mathbf{I}
	\right]\,.
	\label{eqn:Lambda}
\end{align}
%

From the parentheses of the second line in Eq.~\eqref{eqn:Z}, we obtain the inverse of the full graphene Green's function, given by
%
\begin{align}
	 \mathcal{G}_{z} & =\left[\left(G_{z}^0\right)^{-1} - \frac{1}{N} \Theta^\dagger\mathbf{I}\left(\Delta + V\Gamma_{0,z}V^\dagger
	 \right)\mathbf{I}^T\Theta
	 \right]^{-1}=
	 G_{z}^0+
	  \frac{1}{N} G_{z}^0\Theta^\dagger\mathbf{I} 
	 D_{z}
	 \mathbf{I}^T\Theta
	 G_{z}^0\,,
	 \label{eqn:Full_G}
	 \\
	 D_{z} &=\left[
	\left(\Delta + V\Gamma_{0,z}V^\dagger
	 \right)^{-1}
	 -
	\mathbf{I}^T
	\boldsymbol{\Xi}_{z} \mathbf{I}
	 \right]^{-1}\,.
	 \label{eqn:D}
\end{align}
%

Using Eq.~\eqref{eqn:Full_G}, it is possible to calculate the real-space graphene Green's function $\mathcal{G}_{i\omega_n + \mu,\mathbf{R}}^s = N^{-1}\sum_{\mathbf{qq}'} \langle \bar{\psi}^s_{\omega_n\mathbf{q}}\psi^s_{\omega_n\mathbf{q}'}\rangle e^{i\left(\mathbf{q}'-\mathbf{q}\right)\cdot\mathbf{R}}$, where $s$ denotes the sublattice and the correlation functions are the diagonal elements of the $\left[\mathcal{G}_{i\omega_n+\mu} \right]_{\mathbf{q}'\mathbf{q}}$ blocks:
%
\begin{align}
	\mathcal{G}_{z,\mathbf{R}} 
	 &=\Xi^\mathbf{0}_{z}
	 +
	\sum_{jk}
	\Xi_{z}^{\mathbf{R} - \mathbf{R}_j}
	  \left(\mathbf{I} 
	 D_{z}
	 \mathbf{I}^T
	 \right)_{jk}
	 \Xi_{z}^{\mathbf{R}_k - \mathbf{R}}
	 =\Xi^\mathbf{0}_{z}
	 +
	 \begin{pmatrix}
	 	\Xi_{z}^{\mathbf{R} - \mathbf{R}_1} & \cdots
	 \end{pmatrix}
	 \mathbf{I} 
	 D_{z}
	 \mathbf{I}^T
	 \begin{pmatrix}
	 	\Xi_{z}^{\mathbf{R}_1 - \mathbf{R}} \\ \vdots
	 \end{pmatrix}\,.
	 \label{eqn:Full_G_Real}
\end{align}
%

\section{Occupation Number}

The local density at $s$ sublattice at the unit cell at $\mathbf{R}$ is given by $\rho_{\sigma,\mathbf{R}} = \beta^{-1}\sum_{\omega_n}s^T \mathcal{G}_{i\omega_n + \mu,\sigma,\mathbf{R}}s$, where $s^T = \begin{pmatrix}
	1 &0
\end{pmatrix}$ or $\begin{pmatrix}
	0&1
\end{pmatrix}$. Since we are interested in variation of the charge density due to the perturbations of the pristine system, we concentrate on the second term of Eq.~\eqref{eqn:Full_G_Real} and define the perturbation-induced correction to the density
%
\begin{align}
	\delta \rho_{\mathbf{R}}^s
	 & =
    \frac{1}{\beta}\sum_{\omega_n}
	s^T\begin{pmatrix}
	 	\Xi_{i\omega_n + \mu}^{\mathbf{R} - \mathbf{R}_1} & \cdots
	 \end{pmatrix}
	 \mathbf{I} 
	 D_{i\omega_n + \mu}
	 \mathbf{I}^T
	 \begin{pmatrix}
	 	\Xi_{i\omega_n + \mu}^{\mathbf{R}_1 - \mathbf{R}} \\ \vdots
	 \end{pmatrix}
	s
	\,.
	\label{eqn:Delta_rho}
\end{align}
%

Similarly, it is possible to calculate the impurity occupation number from Eq.~\eqref{eqn:Gamma}:
%
\begin{equation}
    \rho_{k}= \frac{1}{\beta}\sum_{\omega_n} \Gamma_{i\omega_n + \mu, k}  \,.
    \label{eqn:rho_imp}
\end{equation}
%


By taking the $k$th diagonal entry of $-2\mathrm{Im}\left[\Gamma_{\omega + i0^+}\right]$ and $-2\mathrm{Im}\left[\mathcal{G}_{\omega+i0^+,\mathbf{R}}^s\right]$, we obtain the spectral functions for the $k$th impurity and the corresponding carbon atom, respectively. 

At zero temperature, the summation over the Matsubara frequencies can be replaced by an integral: $\beta^{-1}\sum_{\omega_n} f(i\omega_n + \mu)\rightarrow (2\pi)^{-1} \int d\omega f(i\omega + \mu)$. For finite temperatures, we get $\beta^{-1}\sum_{\omega_n} f(i\omega_n + \mu)\rightarrow -2 \mathrm{Im}\left[\int d\omega f(\omega + i0 + \mu)n_F(\omega)\right]$, where $n_F(\omega)$ is the Fermi-Dirac distribution function.

\section{Graphene Propagator}
\label{sec:Propagator}

To compute $\Xi_z^\mathbf{R}$, we first introduce
%
\begin{equation}
	\Omega^{u,v}_z =
	\frac{1}{N}\sum_{\mathbf{q}\in\mathrm{BZ}}
	\frac{
		e^{i\mathbf{q}\cdot \left(u\mathbf{d}_1 + v\mathbf{d}_2\right)}
	}
	{z^2 - t^2\left| f_{1,\mathbf{q}}\right|^2}
	\label{eqn:Omega_R}
\end{equation}
%
with $ u\mathbf{d}_1 + v\mathbf{d}_2 = \frac{d}{2}\left(u - v, \sqrt{3}\left(u+v\right)\right)$ and $t = 2.8$~eV as the nearest-neighbor hopping energy. Using $\mathbf{q}\cdot \left(u\mathbf{d}_1 + v\mathbf{d}_2\right)  = \frac{d}{2}\left[\left(u - v\right)q_x + \sqrt{3}\left(u+v\right)q_y\right]$ and turning the momentum sum into an integral yields
%
\begin{align}
	\Omega^{u,v}\left(z\right) 
	& = \frac{1}{\left(2\pi\right)^2}\oint dx \oint dy
	\frac
	{e^{i \left[\left(u - v\right)x + \left(u+v\right)y\right]}}
	{z^2 - t^2\left(1 + 4\cos^2 x + 4 \cos x\cos y \right)}\,.
	\label{eqn:Omega_R_2}
\end{align}
%
From
%
\begin{equation}
	\oint d\theta \frac{e^{il\theta}}{w-\cos\theta} = 2\pi \frac{\left(w - \sqrt{w - 1}\sqrt{w + 1}\right)^{|l|}}{\sqrt{w - 1}\sqrt{w + 1}}\,,
	\label{eqn:Ang_Int}
\end{equation}
%
we get
%
\begin{align}
	\Omega^{u,v}_z &= \frac{1}{2\pi}\frac{1}{4t^2}
	\oint dx \frac{e^{i\left(u - v\right)x}}{\cos x}\frac{\left(W - \sqrt{W - 1}\sqrt{W + 1}\right)^{|u+v|}}{\sqrt{W - 1}\sqrt{W + 1}}\,,
	\label{eqn:Omega_R_3}
	\\
	W &= \frac{\frac{z^2}{t^2}-1}{4\cos x}-\cos x\,.
	\label{eqn:W}
\end{align}
%
Finally, $\Xi^{\mathbf{R}}_z$ for $\mathbf{R} = u\mathbf{d}_1 + v\mathbf{d}_2$ can be written as
%
\begin{align}
	\Xi^{\mathbf{R}}_z 
	&=
	\begin{pmatrix}
		z\Omega^{u,v}_z
		&
		- t\left[\Omega^{u,v}_z + \Omega^{u,v}_{+,z} \right]
		\\
		- t\left[\Omega^{u,v}_z + \Omega^{u,v}_{-,z}\right]
		&
		z\Omega^{u,v}_z
	\end{pmatrix}\,,
	\\
	\Omega^{u,v}_{\pm,z}
	&= 
	 \frac{1}{2\pi}\frac{1}{4t^2}
	\oint dx \,2e^{i\left(u - v\right)x}\frac{\left(W - \sqrt{W - 1}\sqrt{W + 1}\right)^{|u+v\pm 1|}}{\sqrt{W - 1}\sqrt{W + 1}}
\,.
\end{align}
%



Starting with the pristine graphene Hamiltonian, we introduce the external electric charge by including a position-dependent potential

%
\begin{align}
    \hat{H} &= \sum_{\mathbf{q}\sigma} c^\dagger_{\mathbf{q}\sigma}
    H_{0,\mathbf{q}}^G 
    c_{\mathbf{q}\sigma}
    +\sum_{\mathbf{R}\sigma} c_{\mathbf{R}\sigma}^\dagger \Delta_{\mathbf{R}} c_{\mathbf{R}\sigma}
    \,.
    \label{eqn:H_0}
\end{align}
%
Here, $H_{0,\mathbf{q}}^G$ is the pristine Hamiltonian matrix and $c_{\mathbf{q}\sigma}^\dagger = \begin{pmatrix} a_{\mathbf{q}\sigma}^\dagger&b_{\mathbf{q}\sigma}^\dagger
\end{pmatrix}$ is a vector of creation operators in the momentum space with $c_{\mathbf{R}\sigma}^\dagger$ as its real-space counterpart and $\sigma$ as the spin index. $\Delta_{\mathbf{R}}$ is a diagonal $2\times 2$ matrix of potential energies for the two sublattice atoms for the unit cell at $\mathbf{R}$ computed from

%
\begin{equation}
    \Delta_\mathbf{R}^{A/B} = e\int d\mathbf{r} \,V(\mathbf{r}) \Psi_{A/B,\mathbf{R}}^2(\mathbf{r})\,,
\end{equation}
%
where $\Psi_{A/B,\mathbf{R}}(\mathbf{r})$ is the wave function of the carbon $p_z$ orbital belonging to the $A/B$ sublattice of the unit cell at $\mathbf{R}$ and $V(\mathbf{r})$ is an external potential.

To include screening, we introduce electron-electron interaction. Assuming that the system is not magnetized so that the densities of spin-up and spin-down electrons are equal, the mean-field expression for the electronic interaction becomes

%
\begin{align}
    \hat{H}_{ee} &= 
    2\sum_{\mathbf{R}\sigma i}
    c^\dagger_{\mathbf{R}\sigma i} 
    \left(\sum_{\mathbf{R}'j}
    U_{\mathbf{RR}'}^{ij}
    \langle n_{\mathbf{R}'}^j\rangle
    \right)
    c_{\mathbf{R}\sigma i} \,,
    \label{eqn:H_ee}
\end{align}
%
where $\langle n_\mathbf{R}^j\rangle$ the mean-field occupation number for sublattice $j$ at the unit cell $\mathbf{R}$, 

%
\begin{align}
    U_{\mathbf{R}\mathbf{{R}'}}^{ij} = 
    \frac{1}{2^{\delta_{ij}\delta_{\mathbf{RR}'}}}
    \int d\mathbf{r}\, d\mathbf{r}' \frac{e^2}{|\mathbf{r} - \mathbf{r}'|}
    \Psi_{i,\mathbf{R}}^2(\mathbf{r})
    \Psi_{j,\mathbf{R}'}^2(\mathbf{r}')
    \label{eqn:U}
\end{align}
%
is the Coulomb interaction, and the subscript $i$ labels the sublattice. The prefactor 2 in Eq.~\eqref{eqn:H_ee} accounts for two spins with equal occupation numbers at each $p_z$ orbital. The prefactor $1/2$ in Eq.~\eqref{eqn:U} for $i = j$ and $\mathbf{R} = \mathbf{R}'$ guarantees that an electron of interacts only with one electron of the opposite spin at its own location.

In the absence of the external perturbation, the occupation number is the same for all the orbitals in the system. Denoting this number by $n^0$, we can write

%
\begin{align}
    \hat{H}_{ee} &= 
   2\sum_{\mathbf{R}\sigma i}
    c^\dagger_{\mathbf{R}\sigma i} 
    \left[\sum_{\mathbf{R}'j}
    U_{\mathbf{RR}'}^{ij}
   \left( \langle n_{\mathbf{R}'}^j\rangle - n^0\right)
    \right]
    c_{\mathbf{R}\sigma i} 
    \nonumber
    \\
    &+
    2\sum_{\mathbf{R}\sigma i}
    c^\dagger_{\mathbf{R}\sigma i} 
    \left(\sum_{\mathbf{R}'j}
    U_{\mathbf{RR}'}^{ij}
    n^0
    \right)
    c_{\mathbf{R}\sigma i} \,,
    \label{eqn:H_ee_2}
\end{align}
%
where second term can be absorbed into the chemical potential. The first term indicates that only the orbitals whose occupancy differs significantly from $n^0$ contribute to the electron-electron interaction.

Combining the first term of Eq.~\eqref{eqn:H_ee_2} with the external potential allows us to define the occupancy-dependent screened potential
%
\begin{equation}
    \tilde{\Delta}_\mathbf{R}^i = \Delta_\mathbf{R}^i + 2\sum_{\mathbf{R}'j}
    U_{\mathbf{RR}'}^{ij}
   \left( \langle n_{\mathbf{R}'}^j\rangle - n^0\right)
\end{equation}
%
and, consequently,

%
\begin{align}
    \hat{H} &= \sum_{\mathbf{q}\sigma} c^\dagger_{\mathbf{q}\sigma}
    H_{0,\mathbf{q}}^G 
    c_{\mathbf{q}\sigma}
    +\sum_{\mathbf{R}\sigma} c_{\mathbf{R}\sigma}^\dagger \tilde{\Delta}_{\mathbf{R}} c_{\mathbf{R}\sigma}
    \,.
    \label{eqn:H}
\end{align}
%

Using  $c^\dagger_{\mathbf{R}\sigma} = N^{-1/2}\sum_\mathbf{q}c^\dagger_{\mathbf{q}\sigma} e^{-i\mathbf{R}\cdot\mathbf{q}}$, where $N$ is the number of unit cells in the system, one gets

%
\begin{align}
	\sum_{\mathbf{R}\sigma} c_{\mathbf{R}\sigma}^\dagger  \tilde{\Delta}_{\mathbf{R}} c_{\mathbf{R}\sigma}
	=\frac{1}{N}\sum_{\mathbf{qq}'} c_{\mathbf{q}\sigma}^\dagger \Theta_{\mathbf{q}}^\dagger\tilde{\Delta} \Theta_{\mathbf{q}'}c_{\mathbf{q}'\sigma}\,,
	\label{eqn:Delta}
\end{align}
%
where $\Theta_\mathbf{q}$ is a column vector of $\mathbf{1}_{2\times 2} e^{i\mathbf{R}\cdot\mathbf{q}}$ for all $\mathbf{R}$.

When the overlap is included, the two-band eigenvalues are found by solving the generalized eigenvalue problem

%
\begin{align}
	H^G_{0,\mathbf{q}}\Psi_\mathbf{q} = E_\mathbf{q}P_\mathbf{q} \Psi_\mathbf{q}\,,
\end{align}
%
where $P_\mathbf{q}$ is a $2\times 2$ overlap matrix. One can transform this Hamiltonian as follows

%
\begin{align}
	&H^G_{0,\mathbf{q}}\Psi_\mathbf{q} = E_\mathbf{q}P^{1/2}_\mathbf{q} P^{1/2}_\mathbf{q} \Psi_\mathbf{q}
	\nonumber
	\\
	\rightarrow &P^{-1/2}_\mathbf{q}H^G_{0,\mathbf{q}}P^{-1/2}_\mathbf{q}P^{1/2}_\mathbf{q}\Psi_\mathbf{q} = E_\mathbf{q} P^{1/2}_\mathbf{q} \Psi_\mathbf{q}\,.
\end{align}
%

Hence, it is useful to define $c_\mathbf{q} = P_\mathbf{q}^{-1/2}\tilde{c}_\mathbf{q}$ to write the second-quantized Hamiltonian as

%
\begin{align}
    \hat{H} &= \sum_{\mathbf{q}} \tilde{c}_\mathbf{q}^\dagger P_\mathbf{q}^{-1/2}
    H_{0,\mathbf{q}}^G 
    P_\mathbf{q}^{-1/2}\tilde{c}_\mathbf{q}
    \nonumber
    \\
    &+
    \frac{1}{N}\sum_{\mathbf{qq}'} \tilde{c}_\mathbf{q}^\dagger P_\mathbf{q}^{-1/2} \Theta_{\mathbf{q}}^\dagger \tilde{\Delta} \Theta_{\mathbf{q}'}P_{\mathbf{q}'}^{-1/2}\tilde{c}_{\mathbf{q}'}
    \,.
    \label{eqn:H_QFT_2}
\end{align}
%

Next, one can translate the Hamiltonian into the imaginary-time action
%
\begin{align}
    S &= \sum_{\omega_n\mathbf{qq}'} \bar\psi_{\omega_n\mathbf{q}}
    \left[\left(-i\omega_n-\mu\right) \delta_{\mathbf{qq}'}
    + 
    P_\mathbf{q}^{-1/2}
    H^G_{\mathbf{qq}'} 
    P_{\mathbf{q}'}^{-1/2}
    \right]
    \psi_{\omega_n\mathbf{q}'}\,.
    \label{eqn:S}
\end{align}
%
Note that we have combined the $\mathbf{q}$-diagonal and non-diagonal portions of the graphene Hamiltonian into $ H^G_{\mathbf{qq}'}$. Finally, we change the graphene fields back to the original orbitals to get
%
\begin{align}
    S &= \sum_{\omega_n\mathbf{qq}'} \bar\phi_{\omega_n\mathbf{q}}
    \overbrace{
    \left[P_\mathbf{q}\left(-i\omega_n-\mu\right) \delta_{\mathbf{qq}'}
    + 
    H^G_{\mathbf{qq}'} \right]}^{-G^{-1}_{i\omega_n + \mu, \mathbf{qq}'}}
    \phi_{\omega_n\mathbf{q}'}\,.
    \label{eqn:S_2}
\end{align}
%
%Importantly, there is a Jacobian associated 

Defining a pristine graphene Green's function $G_{z}^0 = \left(zP - H^G_0\right)^{-1}$ leads to

%
\begin{align}
	G_{z} &= \left[\left(G_{z}^0\right)^{-1} - \frac{1}{N} \Theta^\dagger \tilde{\Delta}  \Theta\right]^{-1}
	\nonumber
	\\
	&=
	G_{z}^0
	+
	\frac{1}{N} G_{z}^0\Theta^\dagger \tilde{\Delta}
	\left( 1
	-
	 \boldsymbol{\Xi}_{z} \tilde{\Delta}
	\right)^{-1} \Theta G_{z}^0\,,	
\end{align}
%
where $\boldsymbol{\Xi}^0_{z} = \Theta G_{z}^0\Theta^\dagger / N$ with entries $\left[\boldsymbol{\Xi}^0_{z}\right]_{\mathbf{R}_j\mathbf{R}_k} = \Xi_{z}^0\left(\mathbf{R}_j - \mathbf{R}_k\right)$, $\Theta$ as a row vector of $\Theta_\mathbf{q}$, and

%
\begin{equation}
    \Xi_z^0\left(\mathbf{R}\right) = \frac{1}{N}\sum_\mathbf{q}G_{z\mathbf{q}}^0
    e^{i \mathbf{R} \cdot\mathbf{q}}\,.
    \label{eqn:Xi}
\end{equation}
%

To calculate $\Xi_z^0$, note that
%
\begin{align}
   \left[ G_{z\mathbf{q}}^0\right]^{-1} &= 
   z\begin{pmatrix}
       1&P f_\mathbf{q}
       \\
       P f_\mathbf{q}^*&1
   \end{pmatrix} - 
   \begin{pmatrix}
       0&-tf_\mathbf{q}
       \\
       -tf_\mathbf{q}^*&0
   \end{pmatrix} 
   \nonumber
   \\
   &= 
   \begin{pmatrix}
       z&(zP+t) f_\mathbf{q}
       \\
       (zP+t) f_\mathbf{q}^*&z
   \end{pmatrix}\,.
\end{align}
%
In other words, including the overlap integral simply changes $t \rightarrow t + zP$ in the non-overlap propagator, derived below.

Sandwiching $G_z$ by $\Theta/\sqrt{N}$ and $\Theta^\dagger/\sqrt{N}$ gives the real-space Green's function

%
\begin{align}
	\boldsymbol{\Xi}_{z} &= 
		\boldsymbol{\Xi}_{z}^0
	+
		\boldsymbol{\Xi}_{z}^0 \tilde{\Delta}
	\left( 1
	-
	 \boldsymbol{\Xi}_{z}^0 \tilde{\Delta}
	\right)^{-1} \boldsymbol{\Xi}_{z}^0\,.	
\end{align}
%

From the real-space Green's function, we can obtain the correlation functions

%
\begin{align}
    \langle \bar{\phi}_{z,\mathbf{R}_i}^i \phi_{z,\mathbf{R}_j}^j\rangle  = \left[\boldsymbol{\Xi}_{z,\mathbf{R}_j \mathbf{R}_i}\right]_{ji}\,.
\end{align}
%


\section{Occupation Number}

One of the quantities that we can calculate using the presented formalism is the occupation number. Typically, one would write $\langle a_\mathbf{R}^\dagger a_\mathbf{R}\rangle$ or $\langle b_\mathbf{R}^\dagger b_\mathbf{R}\rangle$ to obtain the number of electrons at the two sublattices at unit cell $\mathbf{R}$.

Here, because the orbitals are not orthogonal, $d^\dagger_\mathbf{R}$ and $d_\mathbf{R}$ do not constitute a creation/annihilation pair ($d$ can represent any state). One can show, however, that $d_\mathbf{i}^\dagger$ and $\sum_i P_{ij}d_j$, where $P_{ij}$ is the orbital overlap, \emph{do} obey the appropriate commutation relation. Using the fact that we will include only the nearest-neighbor overlap, we write
%
\begin{align}
	\rho_{\mathbf{R}}^a &= 
	\langle a_\mathbf{R}^\dagger a_\mathbf{R}\rangle 
	+ 
	P\langle a_\mathbf{R}^\dagger b_\mathbf{R}\rangle
	+
	P\langle a_\mathbf{R}^\dagger b_{\mathbf{R}+\mathbf{d}_1}\rangle
	+
	P\langle a_\mathbf{R}^\dagger b_{\mathbf{R}+\mathbf{d}_2}\rangle\,,
	\nonumber
	\\
	\rho_{\mathbf{R}}^b &= 
	\langle b_\mathbf{R}^\dagger b_\mathbf{R}\rangle 
	+ 
	P\langle b_\mathbf{R}^\dagger a_\mathbf{R}\rangle
	+
	P\langle b_\mathbf{R}^\dagger a_{\mathbf{R}-\mathbf{d}_1}\rangle
	+
	P\langle b_\mathbf{R}^\dagger a_{\mathbf{R}-\mathbf{d}_2}\rangle\,.
\end{align}
%




\section{Graphene Propagator}
\label{sec:Propagator}

To compute $\Xi_z^0(\mathbf{R})$, we first introduce
%
\begin{equation}
	\Omega^{u,v}_z =
	\frac{1}{N}\sum_{\mathbf{q}\in\mathrm{BZ}}
	\frac{
		e^{i\mathbf{q}\cdot \left(u\mathbf{d}_1 + v\mathbf{d}_2\right)}
	}
	{z^2 - t^2\left| f_{\mathbf{q}}\right|^2}
	\label{eqn:Omega_R}
\end{equation}
%
with $ u\mathbf{d}_1 + v\mathbf{d}_2 = \frac{d}{2}\left(u - v, \sqrt{3}\left(u+v\right)\right)$ and $t = 2.8$~eV as the nearest-neighbor hopping energy. Using $\mathbf{q}\cdot \left(u\mathbf{d}_1 + v\mathbf{d}_2\right)  = \frac{d}{2}\left[\left(u - v\right)q_x + \sqrt{3}\left(u+v\right)q_y\right]$ and turning the momentum sum into an integral yields
%
\begin{align}
	\Omega^{u,v}\left(z\right) 
	& = \frac{1}{\left(2\pi\right)^2}\oint dx \oint dy
	\frac
	{e^{i \left[\left(u - v\right)x + \left(u+v\right)y\right]}}
	{z^2 - t^2\left(1 + 4\cos^2 x + 4 \cos x\cos y \right)}\,.
	\label{eqn:Omega_R_2}
\end{align}
%
From
%
\begin{equation}
	\oint d\theta \frac{e^{il\theta}}{w-\cos\theta} = 2\pi \frac{\left(w - \sqrt{w - 1}\sqrt{w + 1}\right)^{|l|}}{\sqrt{w - 1}\sqrt{w + 1}}\,,
	\label{eqn:Ang_Int}
\end{equation}
%
we get
%
\begin{align}
	\Omega^{u,v}_z &= \frac{1}{2\pi}\frac{1}{4t^2}
	\oint dx \frac{e^{i\left(u - v\right)x}}{\cos x}\frac{\left(W - \sqrt{W - 1}\sqrt{W + 1}\right)^{|u+v|}}{\sqrt{W - 1}\sqrt{W + 1}}\,,
	\label{eqn:Omega_R_3}
	\\
	W &= \frac{\frac{z^2}{t^2}-1}{4\cos x}-\cos x\,.
	\label{eqn:W}
\end{align}
%
Finally, $\Xi^0_z({\mathbf{R}})$ for $\mathbf{R} = u\mathbf{d}_1 + v\mathbf{d}_2$ can be written as
%
\begin{align}
	\Xi^0_z({\mathbf{R}}) 
	&=
	\begin{pmatrix}
		z\Omega^{u,v}_z
		&
		- t\left[\Omega^{u,v}_z + \Omega^{u,v}_{+,z} \right]
		\\
		- t\left[\Omega^{u,v}_z + \Omega^{u,v}_{-,z}\right]
		&
		z\Omega^{u,v}_z
	\end{pmatrix}\,,
	\\
	\Omega^{u,v}_{\pm,z}
	&= 
	 \frac{1}{2\pi}\frac{1}{4t^2}
	\oint dx \,2e^{i\left(u - v\right)x}\frac{\left(W - \sqrt{W - 1}\sqrt{W + 1}\right)^{|u+v\pm 1|}}{\sqrt{W - 1}\sqrt{W + 1}}
\,.
\end{align}
%

Again, to include the overlap, switch $t\rightarrow t+ zP$.


\end{document}